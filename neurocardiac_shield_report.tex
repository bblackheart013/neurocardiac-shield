\documentclass[12pt,a4paper]{article}
\usepackage[utf8]{inputenc}
\usepackage{amsmath,amssymb}
\usepackage{graphicx}
\usepackage{booktabs}
\usepackage{hyperref}
\usepackage{listings}
\usepackage{geometry}
\usepackage{fancyhdr}
\usepackage{cite}

\geometry{margin=1in}

\pagestyle{fancy}
\fancyhf{}
\rhead{NeuroCardiac Shield}
\lhead{NYU Advanced Project}
\rfoot{Page \thepage}

\title{
    \textbf{NeuroCardiac Shield: A Multi-Modal Physiological Monitoring Platform for Real-Time Cardiovascular-Neurological Risk Assessment}
}
\author{
    Mohd Sarfaraz Faiyaz \\
    \textit{Author} \\
    New York University \\
    \and
    Vaibhav Devram Chandgir \\
    \textit{Contributor} \\
    New York University
}
\date{November 2025}

\begin{document}

\maketitle

\begin{abstract}
This paper presents NeuroCardiac Shield, a comprehensive multi-modal physiological monitoring platform that integrates electroencephalography (EEG) and electrocardiography (ECG) signal acquisition with machine learning-based risk prediction. The system implements an end-to-end pipeline encompassing embedded firmware simulation, cloud-based signal processing, ensemble machine learning inference, and real-time visualization. The platform employs an 8-channel EEG acquisition system following the international 10-20 electrode placement standard and 3-lead ECG monitoring with PQRST morphology analysis. Risk prediction utilizes a weighted ensemble combining XGBoost classifiers (60\%) for interpretable feature-based analysis and bidirectional LSTM networks (40\%) for temporal pattern recognition. The system achieves end-to-end latency under one second with inference times of approximately 80 milliseconds. Designed with IEC 62304 medical device software compliance in mind, the platform serves as a research prototype for cardiovascular-neurological risk assessment. This work demonstrates the feasibility of integrating multi-modal biosignal processing with modern machine learning techniques for real-time health monitoring applications.
\end{abstract}

\section{Introduction}

The convergence of neurological and cardiovascular monitoring presents significant opportunities for comprehensive health assessment. Traditional monitoring systems focus on single modalities, potentially missing critical cross-domain indicators of physiological distress. NeuroCardiac Shield addresses this gap by providing simultaneous brain and heart signal analysis with machine learning-powered risk stratification.

\subsection{Motivation}

Cardiovascular diseases remain the leading cause of global mortality, while neurological disorders affect millions worldwide. The interaction between cardiac and neural systems, particularly through autonomic regulation, suggests that combined monitoring could provide superior diagnostic insights compared to isolated measurements.

\subsection{Objectives}

\begin{itemize}
    \item Develop a multi-modal signal acquisition system supporting EEG and ECG
    \item Implement real-time signal processing pipelines for feature extraction
    \item Design ensemble machine learning models for risk prediction
    \item Create an interactive visualization dashboard for clinical interpretation
    \item Ensure compliance with medical device software standards
\end{itemize}

\section{System Architecture}

\subsection{Overview}

The NeuroCardiac Shield architecture comprises four primary layers:

\begin{enumerate}
    \item \textbf{Firmware Layer}: Embedded signal acquisition and transmission
    \item \textbf{Cloud Backend}: Data ingestion, processing, and API services
    \item \textbf{ML Pipeline}: Feature extraction and risk inference
    \item \textbf{Dashboard}: Real-time visualization interface
\end{enumerate}

\subsection{Firmware Design}

The embedded layer simulates an ARM Cortex-M4 microcontroller (STM32F4) acquiring physiological signals. Key specifications include:

\begin{itemize}
    \item Sampling rate: 250 Hz (Nyquist frequency: 125 Hz)
    \item Packet rate: 10 Hz (25 samples per packet)
    \item Packet size: 569 bytes
    \item Communication: BLE 5.0+ compatible (MTU 600 bytes)
\end{itemize}

The data packet structure is defined as:

\begin{equation}
    P = \{T_{ms}, ID_{pkt}, ID_{dev}, F_{status}, S_{EEG}, S_{ECG}, V_{SpO2}, V_{temp}, A_{xyz}, C_{sum}\}
\end{equation}

where $S_{EEG} \in \mathbb{R}^{8 \times 25}$ represents 8-channel EEG data and $S_{ECG} \in \mathbb{R}^{3 \times 25}$ represents 3-lead ECG data.

\section{Signal Processing}

\subsection{Preprocessing Pipeline}

Digital filtering is applied to remove artifacts and noise:

\textbf{EEG Bandpass Filter (0.5-50 Hz):}
\begin{equation}
    H_{EEG}(z) = \frac{b_0 + b_1z^{-1} + ... + b_nz^{-n}}{1 + a_1z^{-1} + ... + a_nz^{-n}}
\end{equation}

\textbf{ECG Bandpass Filter (0.5-40 Hz):}
\begin{equation}
    H_{ECG}(z) = \frac{b_0 + b_1z^{-1} + ... + b_nz^{-n}}{1 + a_1z^{-1} + ... + a_nz^{-n}}
\end{equation}

Butterworth IIR filters of order 4 are employed for their maximally flat magnitude response.

\textbf{Notch Filter (60 Hz):}
\begin{equation}
    H_{notch}(z) = \frac{1 - 2\cos(\omega_0)z^{-1} + z^{-2}}{1 - 2r\cos(\omega_0)z^{-1} + r^2z^{-2}}
\end{equation}

where $\omega_0 = 2\pi f_{notch}/f_s$ and $r$ determines bandwidth.

\subsection{Feature Extraction}

\subsubsection{EEG Features}

Frequency band powers are computed via Welch's power spectral density estimation:

\begin{equation}
    P_{band} = \int_{f_1}^{f_2} S_{xx}(f) df
\end{equation}

where $S_{xx}(f)$ is the power spectral density and bands are defined as:
\begin{itemize}
    \item Delta: 0.5-4 Hz
    \item Theta: 4-8 Hz
    \item Alpha: 8-13 Hz
    \item Beta: 13-30 Hz
    \item Gamma: 30-100 Hz
\end{itemize}

Spectral entropy quantifies signal complexity:
\begin{equation}
    H_{spectral} = -\sum_{i} p_i \log_2(p_i)
\end{equation}

where $p_i$ is the normalized power in frequency bin $i$.

\subsubsection{Heart Rate Variability (HRV) Features}

Time-domain metrics:
\begin{align}
    SDNN &= \sqrt{\frac{1}{N-1}\sum_{i=1}^{N}(RR_i - \overline{RR})^2} \\
    RMSSD &= \sqrt{\frac{1}{N-1}\sum_{i=1}^{N-1}(RR_{i+1} - RR_i)^2} \\
    pNN50 &= \frac{|\{i : |RR_{i+1} - RR_i| > 50ms\}|}{N-1} \times 100\%
\end{align}

Frequency-domain metrics via power spectral analysis:
\begin{align}
    LF &= \int_{0.04}^{0.15} S_{RR}(f) df \\
    HF &= \int_{0.15}^{0.40} S_{RR}(f) df \\
    \frac{LF}{HF} &= \text{Sympathovagal balance indicator}
\end{align}

\section{Machine Learning Models}

\subsection{XGBoost Classifier}

The XGBoost model operates on 74 hand-crafted features:
\begin{equation}
    \hat{y} = \sum_{k=1}^{K} f_k(x), \quad f_k \in \mathcal{F}
\end{equation}

where $\mathcal{F}$ is the space of regression trees. The objective function:
\begin{equation}
    \mathcal{L} = \sum_{i=1}^{n} l(y_i, \hat{y}_i) + \sum_{k=1}^{K} \Omega(f_k)
\end{equation}

with regularization term $\Omega(f) = \gamma T + \frac{1}{2}\lambda\|w\|^2$.

\subsection{Bidirectional LSTM}

The LSTM processes temporal sequences:
\begin{align}
    f_t &= \sigma(W_f \cdot [h_{t-1}, x_t] + b_f) \\
    i_t &= \sigma(W_i \cdot [h_{t-1}, x_t] + b_i) \\
    \tilde{C}_t &= \tanh(W_C \cdot [h_{t-1}, x_t] + b_C) \\
    C_t &= f_t \odot C_{t-1} + i_t \odot \tilde{C}_t \\
    o_t &= \sigma(W_o \cdot [h_{t-1}, x_t] + b_o) \\
    h_t &= o_t \odot \tanh(C_t)
\end{align}

Bidirectional architecture combines forward and backward passes:
\begin{equation}
    h_t = [\overrightarrow{h_t}; \overleftarrow{h_t}]
\end{equation}

\subsection{Ensemble Strategy}

Final prediction combines both models:
\begin{equation}
    P_{ensemble} = w_{XGB} \cdot P_{XGB} + w_{LSTM} \cdot P_{LSTM}
\end{equation}

where $w_{XGB} = 0.6$ and $w_{LSTM} = 0.4$.

Risk score computation:
\begin{equation}
    R = P_{LOW} \cdot 0 + P_{MEDIUM} \cdot 0.5 + P_{HIGH} \cdot 1.0
\end{equation}

Confidence via entropy:
\begin{equation}
    C = 1 - \frac{H(P)}{H_{max}} = 1 - \frac{-\sum_i P_i \log P_i}{\log K}
\end{equation}

\section{Results}

% Placeholder for result figures
\begin{figure}[h]
    \centering
    \fbox{\parbox{0.8\textwidth}{\centering [EEG Waveform Visualization Placeholder]}}
    \caption{8-channel EEG signal display from dashboard}
    \label{fig:eeg}
\end{figure}

\begin{figure}[h]
    \centering
    \fbox{\parbox{0.8\textwidth}{\centering [ECG PQRST Morphology Placeholder]}}
    \caption{ECG waveform with PQRST complex}
    \label{fig:ecg}
\end{figure}

\begin{figure}[h]
    \centering
    \fbox{\parbox{0.8\textwidth}{\centering [Risk Score Gauge Placeholder]}}
    \caption{Risk assessment gauge indicator}
    \label{fig:risk}
\end{figure}

\subsection{Performance Metrics}

\begin{table}[h]
    \centering
    \caption{System Performance}
    \begin{tabular}{lcc}
        \toprule
        Metric & Target & Achieved \\
        \midrule
        Sampling Rate & 250 Hz & 250 Hz \\
        Packet Latency & <200 ms & ~150 ms \\
        ML Inference & <100 ms & ~80 ms \\
        End-to-End & <1 sec & ~660 ms \\
        \bottomrule
    \end{tabular}
\end{table}

\section{Compliance and Standards}

The system design follows:
\begin{itemize}
    \item IEC 62304: Medical device software lifecycle (Class B)
    \item ISO 13485: Quality management systems
    \item FDA SaMD guidance: Software as Medical Device
    \item HIPAA: PHI handling considerations
    \item GDPR: Data protection readiness
\end{itemize}

\textbf{Note}: This is a research prototype. Clinical deployment requires FDA clearance.

\section{Conclusion}

NeuroCardiac Shield demonstrates the feasibility of multi-modal physiological monitoring with integrated machine learning risk assessment. The system successfully processes EEG and ECG signals in real-time, extracting meaningful features and providing risk stratification through ensemble learning. Future work includes hardware integration, clinical validation, and regulatory submission.

\section*{Acknowledgments}

We acknowledge the scipy.signal library for signal processing algorithms, the MNE-Python community for EEG analysis standards, and the Pan-Tompkins algorithm for QRS detection methodology.

\begin{thebibliography}{9}

\bibitem{pan1985}
J. Pan and W. J. Tompkins,
``A Real-Time QRS Detection Algorithm,''
\textit{IEEE Trans. Biomed. Eng.}, vol. BME-32, no. 3, pp. 230--236, 1985.

\bibitem{welch1967}
P. Welch,
``The use of fast Fourier transform for the estimation of power spectra,''
\textit{IEEE Trans. Audio Electroacoust.}, vol. 15, no. 2, pp. 70--73, 1967.

\bibitem{taskforce1996}
Task Force of the European Society of Cardiology,
``Heart rate variability: standards of measurement, physiological interpretation and clinical use,''
\textit{Circulation}, vol. 93, no. 5, pp. 1043--1065, 1996.

\bibitem{iec62304}
IEC 62304:2006,
``Medical device software -- Software life cycle processes,''
International Electrotechnical Commission, 2006.

\bibitem{xgboost}
T. Chen and C. Guestrin,
``XGBoost: A Scalable Tree Boosting System,''
\textit{Proc. 22nd ACM SIGKDD}, pp. 785--794, 2016.

\bibitem{lstm}
S. Hochreiter and J. Schmidhuber,
``Long Short-Term Memory,''
\textit{Neural Computation}, vol. 9, no. 8, pp. 1735--1780, 1997.

\end{thebibliography}

\end{document}
